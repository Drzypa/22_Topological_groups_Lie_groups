\documentclass[12pt]{article}
\usepackage{pmmeta}
\pmcanonicalname{SorgenfreyHalfopenPlane}
\pmcreated{2014-11-06 13:51:15}
\pmmodified{2014-11-06 13:51:15}
\pmowner{rspuzio}{6075}
\pmmodifier{pahio}{2872}
\pmtitle{Sorgenfrey half-open plane}
\pmrecord{7}{36454}
\pmprivacy{1}
\pmauthor{rspuzio}{2872}
\pmtype{Definition}
\pmcomment{trigger rebuild}
\pmclassification{msc}{22-00}
\pmclassification{msc}{55-00}
\pmclassification{msc}{54-00}
\pmsynonym{Sorgenfrey's half-open square topology}{SorgenfreyHalfopenPlane}
\pmsynonym{Sorgenfrey plane}{SorgenfreyHalfopenPlane}

% this is the default PlanetMath preamble.  as your knowledge
% of TeX increases, you will probably want to edit this, but
% it should be fine as is for beginners.

% almost certainly you want these
\usepackage{amssymb}
\usepackage{amsmath}
\usepackage{amsfonts}

% used for TeXing text within eps files
%\usepackage{psfrag}
% need this for including graphics (\includegraphics)
%\usepackage{graphicx}
% for neatly defining theorems and propositions
%\usepackage{amsthm}
% making logically defined graphics
%%%\usepackage{xypic}

% there are many more packages, add them here as you need them

% define commands here
\begin{document}
The \emph{Sorgenfrey plane} is the product of the Sorgenfrey line with itself.  This topology can also be described as the topology on $\mathbb{R}^2$ which arises from the basis $\{ [a,b) \times [c,d) \mid a,b,c,d \in \mathbb{R}, a < b , c < d \}$.

It is interesting to note that, even though the Sorgenfrey line 
enjoys the \PMlinkname{Lindel\"of property}{lindelofspace}, the 
Sorgenfrey plane does not.  To see this, one can note that the 
line $x + y = 0$ is a closed subset in this topology and that 
the induced topology on this line is the discrete topology.  
Since the Lindel\"of property is weakly hereditary, the 
discrete topology on the real line would have to be Lindel\"of 
if the Sorgenfrey plane topology were Lindel\"of.  However, the 
discrete topology on an uncountable set can never have the 
Lindel\"of property, so the Sorgenfrey topology cannot have 
this property either.

{\bf Reference}

Sorgenfrey, R. H. \emph{On the Topological Product of Paracompact Spaces}, Bulletin of the American Mathematical Society, (1947) 631-632
%%%%%
%%%%%
\end{document}
