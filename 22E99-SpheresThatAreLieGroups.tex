\documentclass[12pt]{article}
\usepackage{pmmeta}
\pmcanonicalname{SpheresThatAreLieGroups}
\pmcreated{2013-03-22 17:50:02}
\pmmodified{2013-03-22 17:50:02}
\pmowner{asteroid}{17536}
\pmmodifier{asteroid}{17536}
\pmtitle{spheres that are Lie groups}
\pmrecord{7}{40304}
\pmprivacy{1}
\pmauthor{asteroid}{17536}
\pmtype{Corollary}
\pmcomment{trigger rebuild}
\pmclassification{msc}{22E99}
\pmclassification{msc}{57T10}

\endmetadata

% this is the default PlanetMath preamble.  as your knowledge
% of TeX increases, you will probably want to edit this, but
% it should be fine as is for beginners.

% almost certainly you want these
\usepackage{amssymb}
\usepackage{amsmath}
\usepackage{amsfonts}

% used for TeXing text within eps files
%\usepackage{psfrag}
% need this for including graphics (\includegraphics)
%\usepackage{graphicx}
% for neatly defining theorems and propositions
%\usepackage{amsthm}
% making logically defined graphics
%%%\usepackage{xypic}

% there are many more packages, add them here as you need them

% define commands here

\begin{document}
\PMlinkescapephrase{spheres}
\PMlinkescapephrase{satisfy}

{\bf Theorem -} The only \PMlinkname{spheres}{Sphere} that are Lie groups are $S^0$, $S^1$ and $S^3$.

$\,$

{\bf \emph{Proof:}} $\;$ It is known that $S^0$, $S^1$ and $S^3$ have a Lie group \PMlinkescapetext{structure}.

On the other \PMlinkescapetext{side}, we have seen in the \PMlinkname{parent entry}{CohomologyOfCompactConnectedLieGroups} that the \PMlinkname{cohomology groups}{DeRhamCohomology} of a compact connected Lie group $G$ satisfy

\begin{displaymath}
H^1(G;\mathbb{R})= 0 \;\, \Longrightarrow \;\, H^3(G;\mathbb{R}) \neq 0
\end{displaymath}
The result then follows from the fact that the \PMlinkescapetext{cohomology groups} of spheres satisfy $H^1(S^n;\mathbb{R})= 0$ and $H^3(S^n;\mathbb{R})= 0$ for $n \neq 1,3$. $\square$

%%%%%
%%%%%
\end{document}
