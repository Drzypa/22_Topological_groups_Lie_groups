\documentclass[12pt]{article}
\usepackage{pmmeta}
\pmcanonicalname{ExampleOfAnAlexandroffSpaceWhichCannotBeTurnedIntoATopologicalGroup}
\pmcreated{2013-03-22 18:45:46}
\pmmodified{2013-03-22 18:45:46}
\pmowner{joking}{16130}
\pmmodifier{joking}{16130}
\pmtitle{example of an Alexandroff space which cannot be turned into a topological group}
\pmrecord{4}{41543}
\pmprivacy{1}
\pmauthor{joking}{16130}
\pmtype{Example}
\pmcomment{trigger rebuild}
\pmclassification{msc}{22A05}

\endmetadata

% this is the default PlanetMath preamble.  as your knowledge
% of TeX increases, you will probably want to edit this, but
% it should be fine as is for beginners.

% almost certainly you want these
\usepackage{amssymb}
\usepackage{amsmath}
\usepackage{amsfonts}

% used for TeXing text within eps files
%\usepackage{psfrag}
% need this for including graphics (\includegraphics)
%\usepackage{graphicx}
% for neatly defining theorems and propositions
%\usepackage{amsthm}
% making logically defined graphics
%%%\usepackage{xypic}

% there are many more packages, add them here as you need them

% define commands here

\begin{document}
Let $\mathbb{R}$ denote the set of real numbers and $\tau=\{[a,\infty)\ |\ a\in\mathbb{R}\}\cup\{(b,\infty)\ |\ b\in\mathbb{R}\}$. One can easily verify that $(\mathbb{R},\tau)$ is an Alexandroff space.

\textbf{Proposition.} The Alexandroff space $(\mathbb{R},\tau)$ cannot be turned into a topological group.

\textit{Proof.} Assume that $\mathbb{R}=(\mathbb{R},\tau, \circ)$ is a topological group. It is well known that this implies that there is $H\subseteq\mathbb{R}$ which is open, normal subgroup of $\mathbb{R}$. This subgroup ,,generates'' the topology (see the parent object for more details). Thus $H\neq\mathbb{R}$ because $\tau$ is not antidiscrete. Let $g\in\mathbb{R}$ such that $g\not\in H$ (and thus $gH\cap H=\emptyset$). Then $gH$ is again open (because the mapping $f(x)=g\circ x$ is a homeomorphism). But since both $H$ and $gH$ are open, then $gH\cap H\neq\emptyset$. Indeed, every two open subsets in $\tau$ have nonempty intersection. Contradiction, because diffrent cosets are disjoint. $\square$
%%%%%
%%%%%
\end{document}
