\documentclass[12pt]{article}
\usepackage{pmmeta}
\pmcanonicalname{Topologicalalgebra}
\pmcreated{2013-03-22 14:45:38}
\pmmodified{2013-03-22 14:45:38}
\pmowner{HkBst}{6197}
\pmmodifier{HkBst}{6197}
\pmtitle{topological $*$-algebra}
\pmrecord{12}{36402}
\pmprivacy{1}
\pmauthor{HkBst}{6197}
\pmtype{Definition}
\pmcomment{trigger rebuild}
\pmclassification{msc}{22A30}
\pmclassification{msc}{16W80}
\pmclassification{msc}{16W10}
\pmclassification{msc}{46K05}
\pmclassification{msc}{46H35}
\pmsynonym{topological *-algebra}{Topologicalalgebra}
\pmrelated{BanachAlgebra}
\pmrelated{WeakHopfCAlgebra2}
\pmrelated{VonNeumannAlgebra}
\pmdefines{involution $*$-algebra}
\pmdefines{*-algebra}

\endmetadata

% this is the default PlanetMath preamble.  as your knowledge
% of TeX increases, you will probably want to edit this, but
% it should be fine as is for beginners.

% almost certainly you want these
\usepackage{amssymb}
\usepackage{amsmath}
\usepackage{amsfonts}

% used for TeXing text within eps files
%\usepackage{psfrag}
% need this for including graphics (\includegraphics)
%\usepackage{graphicx}
% for neatly defining theorems and propositions
%\usepackage{amsthm}
% making logically defined graphics
%%%\usepackage{xypic}

% there are many more packages, add them here as you need them

% define commands here
\newenvironment{df}[1][]{\par\noindent\textbf{Definition (#1)}}{}
\begin{document}
\PMlinkescapeword{involution}

\begin{df}[Involution]
An involution on an algebra $A$ over an \PMlinkname{involutory field}{InvolutaryRing} $F$ is a map $\cdot^* : A \to A : a \mapsto a^*$ such that for every $\{a, b\} \subset A$ and $\lambda \in F$ we have
\begin{enumerate}
\item $a^{**} = a$,
\item $(ab)^* = b^* a^*$ and
\item $(\lambda a+b)^* = \lambda^*a^* + b^*$, where $\lambda^*$ denotes the \PMlinkname{involution}{InvolutaryRing} of $\lambda$ in $F$.
\end{enumerate}
\end{df}

\begin{df}[$*$-Algebra]
A $*$-algebra is an algebra with an involution.
\end{df}

\begin{df}[Topological $*$-algebra]
A topological $*$-algebra is a $*$-algebra which is also a topological vector space such that its algebra multiplication and involution are continuous.
\end{df}

\subsubsection{Remarks:}
\begin{itemize}
\item $*$-algebras are a particular \PMlinkescapetext{type} of involutory rings.
\item The involutory field $F$ is often taken as $\mathbb{C}$, where the involution is given by complex conjugation. In this case, condition 3 could be rewritten as:

3.$\;(\lambda a +b)^*= \overline{\lambda}a^*+b^*$
\item Banach *-algebras are topological $*$-algebras.
\end{itemize}

%%%%%
%%%%%
\end{document}
