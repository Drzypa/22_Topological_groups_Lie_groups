\documentclass[12pt]{article}
\usepackage{pmmeta}
\pmcanonicalname{ExampleOfAProofUsingNets}
\pmcreated{2013-03-22 17:11:25}
\pmmodified{2013-03-22 17:11:25}
\pmowner{yark}{2760}
\pmmodifier{yark}{2760}
\pmtitle{example of a proof using nets}
\pmrecord{4}{39508}
\pmprivacy{1}
\pmauthor{yark}{2760}
\pmtype{Example}
\pmcomment{trigger rebuild}
\pmclassification{msc}{22A05}
%\pmkeywords{centre}
%\pmkeywords{center}
%\pmkeywords{Hausdorff}
%\pmkeywords{topological group}
%\pmkeywords{closed}

\endmetadata

\usepackage{amsthm}

\newtheorem*{thm*}{Theorem}
\def\closure{\overline}

\begin{document}
\PMlinkescapeword{basic}
\PMlinkescapeword{centre}
\PMlinkescapeword{center}
\PMlinkescapeword{limits}
\PMlinkescapeword{properties}
\PMlinkescapeword{simple}
\PMlinkescapeword{theorem}

In this entry we will give a simple example
of how nets can be used to prove topological theorems.
The proof will make use of some of the basic properties of nets
listed in the \PMlinkname{parent entry}{Net}.

\begin{thm*}
The \PMlinkname{centre}{GroupCentre} of a Hausdorff topological group is closed.
\end{thm*}

{\bf Proof.}
Let $Z$ be the centre of a Hausdorff topological group $G$.
Let $x\in\closure{Z}$.
Then there is a net $(x_\delta)$ in $Z$ such that $x_\delta\to x$.
Let $g\in G$.
By continuity we have $gx_\delta g^{-1}\to gxg^{-1}$.
But $gx_\delta g^{-1}=x_\delta$, so $gx_\delta g^{-1}\to x$.
As $G$ is Hausdorff, these two limits must be the same.
So $gxg^{-1}=x$, that is, $gx=xg$.
Thus $x\in Z$,
and we have shown that $\closure{Z}=Z$, as required.

%%%%%
%%%%%
\end{document}
