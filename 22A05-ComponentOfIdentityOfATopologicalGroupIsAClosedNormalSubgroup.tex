\documentclass[12pt]{article}
\usepackage{pmmeta}
\pmcanonicalname{ComponentOfIdentityOfATopologicalGroupIsAClosedNormalSubgroup}
\pmcreated{2013-03-22 18:01:42}
\pmmodified{2013-03-22 18:01:42}
\pmowner{asteroid}{17536}
\pmmodifier{asteroid}{17536}
\pmtitle{component of identity of a topological group is a closed normal subgroup}
\pmrecord{6}{40546}
\pmprivacy{1}
\pmauthor{asteroid}{17536}
\pmtype{Theorem}
\pmcomment{trigger rebuild}
\pmclassification{msc}{22A05}

% this is the default PlanetMath preamble.  as your knowledge
% of TeX increases, you will probably want to edit this, but
% it should be fine as is for beginners.

% almost certainly you want these
\usepackage{amssymb}
\usepackage{amsmath}
\usepackage{amsfonts}

% used for TeXing text within eps files
%\usepackage{psfrag}
% need this for including graphics (\includegraphics)
%\usepackage{graphicx}
% for neatly defining theorems and propositions
%\usepackage{amsthm}
% making logically defined graphics
%%%\usepackage{xypic}

% there are many more packages, add them here as you need them

% define commands here

\begin{document}
{\bf Theorem -} Let $G$ be a topological group and $e$ its identity element. The connected component of $e$ is a closed normal subgroup of $G$.

$\,$

{\bf \emph{Proof:}} Let $F$ be the connected component of $e$. All components of a topological space are closed, so $F$ is closed.

Let $a \in F$. Since the multiplication and inversion functions in $G$ are continuous, the set $a F^{-1}$ is also connected, and since $e \in aF^{-1}$ we must have $aF^{-1} \subseteq F$. Hence, for every $b \in F$ we have $ab^{-1} \in F$, i.e. $F$ is a subgroup of $G$.

If $g$ is an arbitrary element of $G$, then $g^{-1}Fg$ is a connected subset containing $e$. Hence $g^{-1}Fg \subset F$ for every $g \in G$, i.e. $F$ is a normal subgroup. $\square$
%%%%%
%%%%%
\end{document}
