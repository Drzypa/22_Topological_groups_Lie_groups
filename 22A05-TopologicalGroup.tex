\documentclass[12pt]{article}
\usepackage{pmmeta}
\pmcanonicalname{TopologicalGroup}
\pmcreated{2013-03-22 15:47:09}
\pmmodified{2013-03-22 15:47:09}
\pmowner{yark}{2760}
\pmmodifier{yark}{2760}
\pmtitle{topological group}
\pmrecord{29}{37744}
\pmprivacy{1}
\pmauthor{yark}{2760}
\pmtype{Definition}
\pmcomment{trigger rebuild}
\pmclassification{msc}{22A05}
%\pmkeywords{group}
%\pmkeywords{topology}
\pmrelated{Group}
\pmrelated{TopologicalSpace}
\pmdefines{homomorphism of topological groups}
\pmdefines{isomorphism of topological groups}

\usepackage{amsfonts}

\def\R{\mathbb{R}}
\def\Z{\mathbb{Z}}

\begin{document}
\PMlinkescapephrase{completely regular}
\PMlinkescapeword{continuous}
\PMlinkescapeword{equivalent}
\PMlinkescapeword{homogeneity}
\PMlinkescapeword{homogeneous}
\PMlinkescapeword{natural}
\PMlinkescapeword{normal}
\PMlinkescapeword{operations}
\PMlinkescapeword{properties}
\PMlinkescapeword{quotient}
\PMlinkescapeword{quotients}
\PMlinkescapeword{section}
\PMlinkescapeword{structure}
\PMlinkescapeword{subgroup}

\section*{Definitions}

A \emph{topological group} is a group $G$ endowed with a topology
such that the multiplication and inverse operations of $G$
are \PMlinkname{continuous}{Continuous}.
That is, the map $G\times G\to G$ defined by $(x,y)\mapsto xy$ is continuous,
where the topology on $G\times G$ is the product topology,
and the map $G\to G$ defined by $x\mapsto x^{-1}$ is also continuous.

Many authors require the topology on $G$ to be Hausdorff,
which is equivalent to requiring that the trivial subgroup be a closed set.

A topology on a group $G$ that makes $G$ into a topology group
is called a \emph{group topology} for $G$.

\section*{Examples}

Any group becomes a topological group if it is given the discrete topology.

Any group becomes a topological group if it is given the indiscrete topology.

The real numbers with the standard topology form a topological group.
More generally, an ordered group with its \PMlinkescapetext{order topology}
is a topological group.

Lie groups are topological groups with additional structure.

Profinite groups are another important class of topological groups;
they arise, for example, in infinite Galois theory.

\section*{Subgroups, quotients and products}

Every \PMlinkname{subgroup}{Subgroup} of a topological group
either has empty interior or is clopen.
In particular, all proper subgroups of a connected topological group
have empty interior.
The closure of any subgroup is also a subgroup,
and the closure of a normal subgroup is normal
(for proofs, see the entry
``\PMlinkname{closure of sets closed under a finitary operation}{ClosureOfSetsClosedUnderAFinitaryOperation}'').
A subgroup of a topological group is itself a topological group,
with the subspace topology.

If $G$ is a topological group and $N$ is a normal subgroup of $G$,
then the quotient group $G/N$ is also a topological group,
with the quotient topology.
This quotient $G/N$ is Hausdorff if and only if $N$ is a closed subset of $G$.

If $(G_i)_{i\in I}$ is a family of topological groups,
then the unrestricted direct product $\prod_{i\in I}G_i$
is also a topological group, with the product topology.

\section*{Morphisms}

Let $G$ and $H$ be topological groups, and let $f\colon G\to H$ be a function.

The function $f$ is said to be a \emph{homomorphism of topological groups}
if it is a group homomorphism and is also continuous.
It is said to be an \emph{isomorphism of topological groups}
if it is both a group isomorphism and a homeomorphism.

Note that it is possible for $f$ to be a continuous group isomorphism
(that is, a bijective homomorphism of topological groups)
and yet not be an isomorphism of topological groups.
This occurs, for example, if $G$ is $\R$ with the discrete topology,
and $H$ is $\R$ with its usual topology,
and $f$ is the identity map on $\R$.

\section*{Topological properties}

While every group can be made into a topological group,
the same cannot be said of every topological space.
In this section we mention some of the properties
that the underlying topological space must have.

Every topological group is bihomogeneous
and \PMlinkname{completely regular}{Tychonoff}.
Note that our earlier claim that a topological group
is Hausdorff if and only if its trivial subgroup is closed
follows from this:
if the trivial subgroup is closed,
then homogeneity ensures that all singletons are closed,
and so the space is \PMlinkname{$\hbox{T}_1$}{T1Space},
and being completely regular is therefore Hausdorff.
A topological group is not necessarily \PMlinkid{normal}{1530}, however,
a counterexample being the unrestricted direct product
of uncountably many copies of the discrete group $\Z$.

Every topological group is obviously an H-space.
Consequently, the fundamental group of a topological group is abelian.
Note that because topological groups are homogeneous,
the fundamental group does not depend (up to isomorphism)
on the choice of basepoint.

Every locally compact topological group
is \PMlinkid{normal}{1530} and strongly paracompact.

Every connected locally compact topological group is $\sigma$-compact.

\section*{Other notes}

Every topological group possesses a natural uniformity,
which induces the topology.
See the entry about the \PMlinkname{uniformity of a topological group}{UniformStructureOfATopologicalGroup}.

A locally compact topological group
possesses a natural measure, called the Haar measure.
%%%%%
%%%%%
\end{document}
