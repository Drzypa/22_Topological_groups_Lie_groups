\documentclass[12pt]{article}
\usepackage{pmmeta}
\pmcanonicalname{O2}
\pmcreated{2013-03-22 17:57:38}
\pmmodified{2013-03-22 17:57:38}
\pmowner{rspuzio}{6075}
\pmmodifier{rspuzio}{6075}
\pmtitle{O(2)}
\pmrecord{8}{40461}
\pmprivacy{1}
\pmauthor{rspuzio}{6075}
\pmtype{Example}
\pmcomment{trigger rebuild}
\pmclassification{msc}{22E10}
\pmclassification{msc}{22E15}

% this is the default PlanetMath preamble.  as your knowledge
% of TeX increases, you will probably want to edit this, but
% it should be fine as is for beginners.

% almost certainly you want these
\usepackage{amssymb}
\usepackage{amsmath}
\usepackage{amsfonts}

% used for TeXing text within eps files
%\usepackage{psfrag}
% need this for including graphics (\includegraphics)
%\usepackage{graphicx}
% for neatly defining theorems and propositions
%\usepackage{amsthm}
% making logically defined graphics
%%%\usepackage{xypic}

% there are many more packages, add them here as you need them

% define commands here

\begin{document}
\textbf{still being written}

An elementary example of a Lie group is afforded by O(2),
the orthogonal group in two dimensions.  This is the set
of transformations of the plane which fix the origin and 
preserve the distance between points.  It may be shown 
that a transform has this property if and only if it is of
the form
\[
 \begin{pmatrix} x \\ y \end{pmatrix} \mapsto 
 M \begin{pmatrix} x \\ y \end{pmatrix} ,
\]
where $M$ is a $2 \times 2$ matrix such that $M^T M = I$.
(Such a matrix is called orthogonal.)

It is easy enough to check that this is a group.  To see
that it is a Lie group, we first need to make sure that it
is a manifold.  To that end, we will parameterize it.
Calling the entries of the matrix $a,b,c,d$, the condition
 becomes
\[
 \begin{pmatrix} 0 & 1 \\ 1 & 0 \end{pmatrix} =
 \begin{pmatrix} a & b \\ c & d \end{pmatrix}^T
 \begin{pmatrix} a & b \\ c & d \end{pmatrix} =
 \begin{pmatrix} a^2 + c^2 & ab + cd \\
       ab + cd & b^2 + d^2 \end{pmatrix}
\]
which is equivalent to the following system of equations:
\begin{align*}
 a^2 + c^2 &= 1 \\
 ab + cd &= 0 \\
 b^2 + d^2 &= 1
\end{align*}
The first of these equations can be solved by introducing a
parameter $\theta$ and writing $a = \cos \theta$ and $c =
\sin \theta$.  Then the second equation becomes $b \cos
\theta + d \sin \theta = 0$, which can be solved by 
introducing a parameter $r$:
\begin{align*}
 b &= - r \sin \theta \\
 d &= r \cos \theta
\end{align*}
Substituting this into the third equation results in $r^2 = 1$,
so $r = -1$ or $r = +1$.  This means we have two matrices for
each value of $\theta$:
\[
 \begin{pmatrix} \cos \theta & - \sin \theta \\
                 \sin \theta &   \cos \theta
 \end{pmatrix} \qquad
 \begin{pmatrix} \cos \theta &   \sin \theta \\
                 \sin \theta & - \cos \theta
 \end{pmatrix}
\]

Since more than one value of $\theta$ will produce the same
matrix, we must restrict the range in order to obtain a 
bona fide coordinate.  Thus, we may cover $O(2)$ with an
atlas consisting of four neighborhoods:
\begin{align*}
\left\{  \begin{pmatrix} \cos \theta & - \sin \theta \\
                 \sin \theta &   \cos \theta
         \end{pmatrix} \mid
         - {3 \over 4} \pi < \theta <  {3 \over 4} \pi
\right\} \\
\left\{  \begin{pmatrix} \cos \theta & - \sin \theta \\
                 \sin \theta &   \cos \theta
         \end{pmatrix} \mid
         {1 \over 4} \pi < \theta <  {7 \over 4} \pi
\right\} \\
\left\{  \begin{pmatrix} \cos \theta &  \sin \theta \\
                 \sin \theta & - \cos \theta
         \end{pmatrix} \mid
         - {3 \over 4} \pi < \theta <  {3 \over 4} \pi
\right\} \\
\left\{  \begin{pmatrix} \cos \theta &  \sin \theta \\
                 \sin \theta & - \cos \theta
         \end{pmatrix} \mid
         {1 \over 4} \pi < \theta <  {7 \over 4} \pi
\right\}
\end{align*}
Every element of $O(2)$ must belong to at least one of
these neighborhoods.  It its trivial to check that the
transition functions between overlapping coordinate
patches are 
%%%%%
%%%%%
\end{document}
