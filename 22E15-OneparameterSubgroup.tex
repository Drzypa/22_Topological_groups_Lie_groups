\documentclass[12pt]{article}
\usepackage{pmmeta}
\pmcanonicalname{OneparameterSubgroup}
\pmcreated{2013-03-22 14:54:01}
\pmmodified{2013-03-22 14:54:01}
\pmowner{CWoo}{3771}
\pmmodifier{CWoo}{3771}
\pmtitle{one-parameter subgroup}
\pmrecord{7}{36583}
\pmprivacy{1}
\pmauthor{CWoo}{3771}
\pmtype{Definition}
\pmcomment{trigger rebuild}
\pmclassification{msc}{22E15}
\pmclassification{msc}{22E10}
\pmsynonym{1-parameter subgroup}{OneparameterSubgroup}

% this is the default PlanetMath preamble.  as your knowledge
% of TeX increases, you will probably want to edit this, but
% it should be fine as is for beginners.

% almost certainly you want these
\usepackage{amssymb,amscd}
\usepackage{amsmath}
\usepackage{amsfonts}

% used for TeXing text within eps files
%\usepackage{psfrag}
% need this for including graphics (\includegraphics)
%\usepackage{graphicx}
% for neatly defining theorems and propositions
%\usepackage{amsthm}
% making logically defined graphics
%%%\usepackage{xypic}

% there are many more packages, add them here as you need them

% define commands here
\begin{document}
Let $G$ be a Lie Group.  A 
\emph{one-parameter subgroup} of $G$ is a group homomorphism $$\phi\colon\mathbb{R}\to G$$ that is also a differentiable 
map at the same time.  We view $\mathbb{R}$ additively and $G$
multiplicatively, so that $\phi(r+s)=\phi(r)\phi(s)$.

\textbf{Examples}.
\begin{enumerate}
\item If $G=\operatorname{GL}(n,k)$, where $k=\mathbb{R}$ or $\mathbb{C}$, then any one-parameter subgroup has the form
$$\phi(t)=e^{tA},$$ where $A=\frac{d\phi}{dt}(0)$ is an $n\times n$ matrix over $k$.  The matrix $A$ is just a 
tangent vector to the Lie group $\operatorname{GL}(n,k)$.  This property establishes the fact that there is a 
one-to-one correspondence between one-parameter subgroups and tangent vectors of $\operatorname{GL}(n,k)$. The same relationship holds for a general Lie group.
The one-to-one correspondence between tangent vectors at the identity (the
Lie algebra) and one-parameter subgroups is established via the exponential
map instead of the matrix exponential.
\item If $G=\operatorname{O}(n,\mathbb{R})\subseteq\operatorname{GL}(n,\mathbb{R})$, the orthogonal group over $R$, then 
any one-parameter subgroup has the same form as in the example above, except that $A$ is skew-symmetric: 
$A^{\operatorname{T}}=-A$.
\item If $G=\operatorname{SL}(n,\mathbb{R})\subseteq\operatorname{GL}(n,\mathbb{R})$, the special linear group over $R$, 
then any one-parameter subgroup has the same form as in the example above, except that $\operatorname{tr}(A)=0$, where 
$\operatorname{tr}$ is the trace operator.
\item If $G=\operatorname{U}(n)=\operatorname{O}(n,\mathbb{C})\subseteq\operatorname{GL}(n,\mathbb{C})$, the unitary 
group over $C$, then any one-parameter subgroup has the same form as in the example above, except that $A$ is \PMlinkname{skew-Hermitian}{SkewHermitianMatrix}: $A=-A^{*}=-\overline{A}^{\operatorname{T}}$ and $\operatorname{tr}(A)=0$.
\end{enumerate}
%%%%%
%%%%%
\end{document}
