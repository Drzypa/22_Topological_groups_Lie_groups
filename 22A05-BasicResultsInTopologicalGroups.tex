\documentclass[12pt]{article}
\usepackage{pmmeta}
\pmcanonicalname{BasicResultsInTopologicalGroups}
\pmcreated{2013-03-22 17:37:38}
\pmmodified{2013-03-22 17:37:38}
\pmowner{asteroid}{17536}
\pmmodifier{asteroid}{17536}
\pmtitle{basic results in topological groups}
\pmrecord{16}{40048}
\pmprivacy{1}
\pmauthor{asteroid}{17536}
\pmtype{Result}
\pmcomment{trigger rebuild}
\pmclassification{msc}{22A05}
\pmrelated{PolishGSpace}
\pmrelated{PolishGroup}

% this is the default PlanetMath preamble.  as your knowledge
% of TeX increases, you will probably want to edit this, but
% it should be fine as is for beginners.

% almost certainly you want these
\usepackage{amssymb}
\usepackage{amsmath}
\usepackage{amsfonts}

% used for TeXing text within eps files
%\usepackage{psfrag}
% need this for including graphics (\includegraphics)
%\usepackage{graphicx}
% for neatly defining theorems and propositions
%\usepackage{amsthm}
% making logically defined graphics
%%%\usepackage{xypic}

% there are many more packages, add them here as you need them

% define commands here

\begin{document}
The purpose of this entry is to list some \PMlinkescapetext{basic} and useful results concerning the topological \PMlinkescapetext{structure} of topological groups. We will use the following notation whenever $A, B$ are subsets of a topological group $G$ and $r$ an element of $G$:
\begin{itemize}
\item $Ar := \{ar: a \in A\}$
\item $rA := \{ra: a \in A\}$
\item $AB:=\{ab: a \in A,\, b \in B\}$
\item $A^2 := \{a_1a_2: a_1,a_2 \in A\}$
\item $A^{-1} := \{a^{-1}: a\in A\}$
\item $\overline{A}$ denotes the closure of $A$
\end{itemize}
$\quad$

{\bf \PMlinkescapetext{Proposition} 1 -} Let $G$ be a topological group and $r \in G$. The left multiplication $s \mapsto rs$, \PMlinkescapetext{right} multiplication $s \mapsto sr$, and inversion $s \mapsto s^{-1}$, are homeomorphisms of $G$.


{\bf \PMlinkescapetext{Proposition} 2 -} Let $G$ be a topological group and $e \in G$ the identity element. Let $\mathcal{B}$ be a neighborhood base around $e$. Then $\{Br\}_{B \in \mathcal{B}}$ is a neighborhood base around $r \in G$ and $\{Br:B\in \mathcal{B} \text{ and }\, r \in G \}$ is a \PMlinkname{basis}{BasisTopologicalSpace} for the topology of $G$.


{\bf \PMlinkescapetext{Proposition} 3 -} Let $G$ be a topological group. If $U \subseteq G$ is open and $V$ is any subset of $G$, then $UV$ is an open set in $G$. 


{\bf \PMlinkescapetext{Proposition} 4 -} Let $G$ be a topological group and $K, L$ compact sets in $G$. Then $KL$ is also compact.


{\bf \PMlinkescapetext{Proposition} 5 -} Let $G$ be a topological group and $e \in G$ the identity element. If $V$ is a neighborhood of $e$ then $V \subset \overline{V} \subset V^2$.


{\bf \PMlinkescapetext{Proposition} 6 -} Let $G$ be a topological group, $e \in G$ the identity element and $W$ a neighborhood around $e$. Then there exists a neighborhood $U$ around $e$ such that $U^2 \subset W$.



{\bf \PMlinkescapetext{Proposition} 7 -} Let $G$ be a topological group, $e \in G$ the identity element and $W$ a neighborhood around $e$. Then there exists a \PMlinkname{symmetric}{SymmetricSet} neighborhood $U$ around $e$ such that $U^2\subseteq W$.

{\bf \PMlinkescapetext{Proposition} 8 -} Let $G$ be a topological group. If $H$ is a subgroup of $G$, then so is $\overline{H}$.

{\bf \PMlinkescapetext{Proposition} 9-} Let $G$ be a topological group. If $H$ is an open subgroup of $G$, then $H$ is also closed.

 

%%%%%
%%%%%
\end{document}
