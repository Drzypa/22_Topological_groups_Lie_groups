\documentclass[12pt]{article}
\usepackage{pmmeta}
\pmcanonicalname{LoopAlgebra}
\pmcreated{2013-03-22 15:30:07}
\pmmodified{2013-03-22 15:30:07}
\pmowner{rspuzio}{6075}
\pmmodifier{rspuzio}{6075}
\pmtitle{loop algebra}
\pmrecord{12}{37361}
\pmprivacy{1}
\pmauthor{rspuzio}{6075}
\pmtype{Definition}
\pmcomment{trigger rebuild}
\pmclassification{msc}{22E60}
\pmclassification{msc}{22E65}
\pmclassification{msc}{22E67}
%\pmkeywords{Kac-Moody Lie algebra}
\pmdefines{loop algebra}

% this is the default PlanetMath preamble.  as your knowledge
% of TeX increases, you will probably want to edit this, but
% it should be fine as is for beginners.

% almost certainly you want these
\usepackage{amssymb}
\usepackage{amsmath}
\usepackage{amsfonts}

% used for TeXing text within eps files
%\usepackage{psfrag}
% need this for including graphics (\includegraphics)
%\usepackage{graphicx}
% for neatly defining theorems and propositions
%\usepackage{amsthm}
% making logically defined graphics
%%%\usepackage{xypic}

% there are many more packages, add them here as you need them

% define commands here
\begin{document}
Let $\mathfrak{g}$ be a Lie algebra over a field $\mathbb{K}$. The \textbf{loop algebra} based on $\mathfrak{g}$ is defined to be $\mathcal{L}(\mathfrak{g}) := \mathfrak{g} \otimes_{\mathbb{K}} \mathbb{K}[t, t^{-1}]$ as a vector space over $\mathbb{K}$. The Lie bracket is determined by

\[ \left[ X \otimes t^k, Y \otimes t^l \right] = \left[ X, Y \right]_{\mathfrak{g}} \otimes t^{k+l} \]

where $\left[\,,\,\right]_{\mathfrak{g}}$ denotes the Lie bracket from $\mathfrak{g}$.

This clearly determines a Lie bracket. For instance the three term sum in the Jacobi identity (for elements which are homogeneous in $t$) simplifies to the three term sum for the Jacobi identity in $\mathfrak{g}$ tensored with a power of $t$ and thus is zero in $\mathcal{L}(\mathfrak{g})$.

The name ``loop algebra'' comes from the fact that this Lie algebra arises in the study of Lie algebras of loop groups.  For the time being, assume that $\mathbb{K}$ is the real or complex numbers so that the familiar structures of analysis and topology are available.  Consider the set of all mappings from the circle $S^1$ (we may think of this circle more concretely as the unit circle of the complex plane) to a finite-dimensional Lie group $G$ with Lie algebra is $\mathfrak{g}$.  We may make this set into a group by defining multiplication pointwise: given $a, b \colon S^1 \to G$, we define $(a \cdot b)(x) = a(x) 
\cdot b(x)$.
\begin{thebibliography}{1}
\bibitem{Kac}
Victor Kac, \emph{Infinite Dimensional Lie Algebras}, Third edition. Cambridge University Press, Cambridge, 1990.
\end{thebibliography}
%%%%%
%%%%%
\end{document}
