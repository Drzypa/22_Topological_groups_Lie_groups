\documentclass[12pt]{article}
\usepackage{pmmeta}
\pmcanonicalname{ModularFunction}
\pmcreated{2013-03-22 17:58:18}
\pmmodified{2013-03-22 17:58:18}
\pmowner{asteroid}{17536}
\pmmodifier{asteroid}{17536}
\pmtitle{modular function}
\pmrecord{8}{40479}
\pmprivacy{1}
\pmauthor{asteroid}{17536}
\pmtype{Definition}
\pmcomment{trigger rebuild}
\pmclassification{msc}{22D05}
\pmclassification{msc}{28C10}
\pmsynonym{Haar modulus}{ModularFunction}
\pmsynonym{modular character}{ModularFunction}
\pmsynonym{modular homomorphism}{ModularFunction}

\endmetadata

% this is the default PlanetMath preamble.  as your knowledge
% of TeX increases, you will probably want to edit this, but
% it should be fine as is for beginners.

% almost certainly you want these
\usepackage{amssymb}
\usepackage{amsmath}
\usepackage{amsfonts}

% used for TeXing text within eps files
%\usepackage{psfrag}
% need this for including graphics (\includegraphics)
%\usepackage{graphicx}
% for neatly defining theorems and propositions
%\usepackage{amsthm}
% making logically defined graphics
%%%\usepackage{xypic}

% there are many more packages, add them here as you need them

% define commands here

\begin{document}
Let $G$ be a locally compact Hausdorff topological group and $\mu$ a left Haar measure. Although left and right Haar measures in $G$ always exist, they generally do not coincide, i.e. a left Haar measure is usually not invariant under right translations. Nevertheless, the right translations of a left Haar measure can be easily described as explained in the following theorem.

$\,$

{\bf Theorem -} Let $G$ be a locally compact Hausdorff topological group and $\mu$ a left Haar measure in $G$. Then, there exists a continuous homomorphism $\Delta:G \longrightarrow \mathbb{R}^+$ such that, for every $t \in G$ and every measurable subset $A$
\begin{displaymath}
\mu(At) = \Delta(t^{-1})\mu(A)
\end{displaymath}
Moreover, if $f:G \longrightarrow \mathbb{C}$ is an integrable function then
\begin{displaymath}
\Delta(t)\int_Gf(st) \mu(s) = \int_G f(s) \mu(s)
\end{displaymath}

$\,$

The function $\Delta$ is called the {\bf modular function} of $G$ (notice that, by uniqueness up to scalar multiple of left Haar measures, $\Delta$ only depends on $G$). Other names for $\Delta$ that can be found are: \emph{Haar modulus}, or \emph{modular character} or \emph{modular homomorphism}.

We now prove the above theorem, except the continuity of $\Delta$ (which is slightly harder to obtain).

$\,$

{\bf \emph{Proof (except continuity of $\Delta$):}}

Let $t \in G$. The function $\nu$, defined on measurable subsets $A$ by
\begin{displaymath}
\nu (A):= \mu(At)
\end{displaymath}

is easily seen to be a measure in $G$. Moreover, $\nu$ is left invariant (since $\mu$ is left invariant) and satisfies the additional conditions to be a left Haar measure. By the uniqueness of left Haar measures, $\mu$ must be a multiple of $\nu$, i.e. $\mu=\Delta(t)\nu$ for some positive scalar $\Delta(t) \in \mathbb{R}^+$. Thus, we have proven that for every measurable subset $A$

\begin{displaymath}
\mu(At)= \Delta(t)^{-1}\mu(A)
\end{displaymath}

Now for $s, t \in G$ we have that $\mu(Ast)=\Delta(st)^{-1}\mu(A)$, but also
\begin{itemize}
\item $\mu(Ast)= \Delta(t)^{-1}\mu(As)$, and
\item $\mu(As) = \Delta(s)^{-1}\mu(A)$
\end{itemize}
So, we can see that, for every measurable subset $A$,

\begin{displaymath}
\Delta(st)^{-1}\mu(A) = \Delta(t)^{-1}\Delta(s)^{-1}\mu(A)
\end{displaymath}
Hence, $\Delta(st) = \Delta(s)\Delta(t)$. Thus, $\Delta$ is an homomorphism.

The statement about integrals of functions follows easily by approximation by simple functions. For simple functions it is easy to see it is true using the now established condition $\mu(At) = \Delta(t^{-1})\mu(A)$. $\square$
%%%%%
%%%%%
\end{document}
