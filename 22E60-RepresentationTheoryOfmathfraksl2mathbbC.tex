\documentclass[12pt]{article}
\usepackage{pmmeta}
\pmcanonicalname{RepresentationTheoryOfmathfraksl2mathbbC}
\pmcreated{2013-03-22 15:30:29}
\pmmodified{2013-03-22 15:30:29}
\pmowner{benjaminfjones}{879}
\pmmodifier{benjaminfjones}{879}
\pmtitle{representation theory of $\mathfrak{sl}_2 \mathbb{C}$}
\pmrecord{7}{37369}
\pmprivacy{1}
\pmauthor{benjaminfjones}{879}
\pmtype{Definition}
\pmcomment{trigger rebuild}
\pmclassification{msc}{22E60}
\pmclassification{msc}{22E47}
%\pmkeywords{special linear Lie algebra}
\pmdefines{sl_2}
\pmdefines{special linear Lie algebra of 2x2 matricies}

% this is the default PlanetMath preamble.  as your knowledge
% of TeX increases, you will probably want to edit this, but
% it should be fine as is for beginners.

% almost certainly you want these
\usepackage{amssymb}
\usepackage{amsmath}
\usepackage{amsfonts}
\usepackage{amsthm}

% used for TeXing text within eps files
%\usepackage{psfrag}
% need this for including graphics (\includegraphics)
%\usepackage{graphicx}
% for neatly defining theorems and propositions
% making logically defined graphics
%%%\usepackage{xypic}

% there are many more packages, add them here as you need them

% define commands here
\begin{document}
The special linear Lie algebra of $2 \times 2$ matricies, denoted by $\mathfrak{sl}_2 \mathbb{C}$, is defined to be the span (over $\mathbb{C}$) of the matricies

\[ E = \left( \begin{array}{cc} 0 & 1 \\ 0 & 0 \end{array} \right),
H = \left( \begin{array}{cc} 1 & 0 \\ 0 & -1 \end{array} \right),
F = \left( \begin{array}{cc} 0 & 0 \\ -1 & 0 \end{array} \right) \]

with Lie bracket given by the commutator of matricies: $[X,Y] := X \cdot Y - Y \cdot X$. The matricies $E, F, H$ satisfy the commutation relations: $[E, F] = H, [H, E] = 2 E, [H, F] = -2 F$.

The representation theory of $\mathfrak{sl}_2 \mathbb{C}$ is a very important tool for understanding the structure theory and representation theory of other Lie algebras (semi-simple finite dimensional Lie algebras, as well as infinite dimensional Kac-Moody Lie algebras).

The finite dimensional, irreducible, representations of $\mathfrak{sl}_2 \mathbb{C}$ are in bijection with the non-negative integers $\mathbb{Z}_{\ge 0}$ as follows. Let $k \in \mathbb{Z}_{\ge 0}$, $V$ be a $\mathbb{C}$-vector space spanned by vectors $v_0, \ldots, v_k$. The following action of $E, H, F$ on $V$ define the unique (up to isomorphism) irreducible representation of $\mathfrak{sl}_2 \mathbb{C}$ of dimension $k+1$ (or of highest weight $k$):

\[ 
\begin{array}{ll} 
E . v_0 & = 0 \\
E . v_i & = (i-1)(k-i+1) v_{i-1} \quad \forall \quad 1 \le i \le k \\
H . v_i & = (k - 2i) v_i         \quad \forall \quad 1 \le i \le k \\ 
F . v_i & = v_{i+1}              \quad \forall \quad 0 \le i < k \\
F . v_k & = 0
\end{array}
\]

The main points are that the one dimensional spaces $\mathbb{C} \cdot v_i$ are eigenspaces for $H$ with eigenvalue $k - 2i$, the operator corresponding to $E$ kills $v_0$ and otherwise sends $\mathbb{C} \cdot v_i \to \mathbb{C} \cdot v_{i-1}$, while $F$ kills $v_k$ and otherwise sends $\mathbb{C} \cdot v_i \to \mathbb{C} \cdot v_{i+1}$. The operator corresponding to $E$ is often called a \emph{raising operator} since it raises the eigenvalue for $H$, and that of $F$ is called a \emph{lowering operator} since it lowers the eigenvalue for $H$.

$\mathfrak{sl}_2 \mathbb{C}$ is a simple Lie algebra, thus by Weyl's Theorem all finite dimensional representations for $\mathfrak{sl}_2 \mathbb{C}$ are completely reducible. So any finite dimensional representation of $\mathfrak{sl}_2 \mathbb{C}$ splits into a direct sum of irreducible representations for various non-negative integers as described above.
%%%%%
%%%%%
\end{document}
