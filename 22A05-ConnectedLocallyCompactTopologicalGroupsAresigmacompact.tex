\documentclass[12pt]{article}
\usepackage{pmmeta}
\pmcanonicalname{ConnectedLocallyCompactTopologicalGroupsAresigmacompact}
\pmcreated{2013-03-22 17:37:12}
\pmmodified{2013-03-22 17:37:12}
\pmowner{asteroid}{17536}
\pmmodifier{asteroid}{17536}
\pmtitle{connected locally compact topological groups are $\sigma$-compact}
\pmrecord{9}{40039}
\pmprivacy{1}
\pmauthor{asteroid}{17536}
\pmtype{Theorem}
\pmcomment{trigger rebuild}
\pmclassification{msc}{22A05}
\pmclassification{msc}{22D05}

\endmetadata

% this is the default PlanetMath preamble.  as your knowledge
% of TeX increases, you will probably want to edit this, but
% it should be fine as is for beginners.

% almost certainly you want these
\usepackage{amssymb}
\usepackage{amsmath}
\usepackage{amsfonts}

% used for TeXing text within eps files
%\usepackage{psfrag}
% need this for including graphics (\includegraphics)
%\usepackage{graphicx}
% for neatly defining theorems and propositions
%\usepackage{amsthm}
% making logically defined graphics
%%%\usepackage{xypic}

% there are many more packages, add them here as you need them

% define commands here

\begin{document}
The main result of this entry is the following theorem (whose proof is given below). The result expressed in the title then follows as a corollary.

{\bf Theorem -} Every locally compact topological group $G$ has an open \PMlinkname{$\sigma$-compact}{SigmaCompact} subgroup $H$.

{\bf Corollary 1 -} Every locally compact topological group is the topological disjoint union of $\sigma$-compact spaces.

{\bf Corollary 2 -} Every connected locally compact topological group is $\sigma$-compact.

We first outline the proofs of the above corollaries:

{\bf \textit{Proof (Corollaries 1 and 2) :}} Let $G$ be a locally compact topological group. The main theorem implies that there is an open $\sigma$-compact subgroup $H$. 

It is known that every open subgroup of $G$ is also closed (see this \PMlinkname{entry}{ClosednessOfSubgroupsOfTopologicalGroups}). Therefore, each $gH$ is a clopen $\sigma$-compact subset of $G$, and $G$ is the topological disjoint union $\displaystyle \bigcup_{g \in G}\; gH$.

Of course, if $G$ is connected then $H$ must be all of $G$. Hence, $G$ is $\sigma$-compact. $\square$

$\quad$

{\bf \textit{Proof (Theorem) :}} Let us fix some notation first. If $A$ is a subset of $G$ we use the notation $A^{-1} := \{ a^{-1} : a \in A\}$,  $A^n : = \{ a_1\dots a_n : a_1, \dots, a_n \in A\}$ and $\overline{A}$ denotes the closure of $A$.

Pick a neighborhood $W$ of $e$ (the identity element of $G$) with compact closure. Then $V:= W \cap W^{-1}$ is a neighborhood of $e$ with compact closure such that $V = V^{-1}$.

Let $ H := \bigcup_{n =1}^{\infty} V^n$. $H$ is clearly a subgroup of $G$. We now only have to prove that $H$ is open and $\sigma$-compact.

We have that (see this \PMlinkname{entry}{BasicResultsInTopologicalGroups} - \PMlinkescapetext{Propositions} 3, 4 and 5)
\begin{itemize}
\item $V^n$ is open
\item $\overline{V}^n$ is compact
\item $\overline{V}^n \subset V^{2n}$
\end{itemize}

So $H$ is open and also $ H = \bigcup_{n =1}^{\infty} \overline{V}^n$, which implies that $H$ is $\sigma$-compact. $\square$
%%%%%
%%%%%
\end{document}
