\documentclass[12pt]{article}
\usepackage{pmmeta}
\pmcanonicalname{Gammasimple}
\pmcreated{2013-03-22 13:52:11}
\pmmodified{2013-03-22 13:52:11}
\pmowner{mathcam}{2727}
\pmmodifier{mathcam}{2727}
\pmtitle{$\Gamma$-simple}
\pmrecord{6}{34612}
\pmprivacy{1}
\pmauthor{mathcam}{2727}
\pmtype{Definition}
\pmcomment{trigger rebuild}
\pmclassification{msc}{22D05}

% this is the default PlanetMath preamble.  as your knowledge
% of TeX increases, you will probably want to edit this, but
% it should be fine as is for beginners.

% almost certainly you want these
\usepackage{amssymb}
\usepackage{amsmath}
\usepackage{amsfonts}

% used for TeXing text within eps files
%\usepackage{psfrag}
% need this for including graphics (\includegraphics)
%\usepackage{graphicx}
% for neatly defining theorems and propositions
%\usepackage{amsthm}
% making logically defined graphics
%%%\usepackage{xypic} 

% there are many more packages, add them here as you need them

% define commands here
\begin{document}
A representation $V$ of $\Gamma$ is \emph{$\Gamma$-simple} if either
\begin{itemize}
\item $V \cong W_1 \oplus W_2$ where $W_1$, $W_2$ are absolutely irreducible for $\Gamma$ and are $\Gamma$-isomorphic, or
\item $V$ is non-absolutely irreducible for $\Gamma$.
\end{itemize}
\cite{1}
\begin{thebibliography}{1}
\bibitem[GSS]{1} Golubitsky, Martin. Stewart, Ian. Schaeffer, G. David.: Singularities and Groups in Bifurcation Theory \textit{(Volume II)}. Springer-Verlag, New York, 1988.
\end{thebibliography}
%%%%%
%%%%%
\end{document}
