\documentclass[12pt]{article}
\usepackage{pmmeta}
\pmcanonicalname{HilbertWeylTheorem}
\pmcreated{2013-03-22 13:39:54}
\pmmodified{2013-03-22 13:39:54}
\pmowner{mathcam}{2727}
\pmmodifier{mathcam}{2727}
\pmtitle{Hilbert-Weyl theorem}
\pmrecord{9}{34323}
\pmprivacy{1}
\pmauthor{mathcam}{2727}
\pmtype{Theorem}
\pmcomment{trigger rebuild}
\pmclassification{msc}{22E20}

\endmetadata

% this is the default PlanetMath preamble.  as your knowledge
% of TeX increases, you will probably want to edit this, but
% it should be fine as is for beginners.

% almost certainly you want these
\usepackage{amssymb}
\usepackage{amsmath}
\usepackage{amsfonts}

% used for TeXing text within eps files
%\usepackage{psfrag}
% need this for including graphics (\includegraphics)
%\usepackage{graphicx}
% for neatly defining theorems and propositions
%\usepackage{amsthm}
% making logically defined graphics
%%%\usepackage{xypic} 

% there are many more packages, add them here as you need them

% define commands here
\begin{document}
\textbf{Theorem:}
Let $\Gamma$ be a compact Lie group acting on $V$.  Then there exists a finite Hilbert basis for the ring $\mathcal{P}(\Gamma )$ \textit{(the set of invariant polynomials)}. \cite{1}

\textbf{proof:}
\begin{quote}
In \cite{1} on page 54.
\end{quote}
\textbf{Theorem:}\textit{(as stated by Hermann Weyl)}
\begin{quote}
The \textit{(absolute)} invariants corresponding to a given set of representations of a finite or a compact Lie group have a finite integrity basis.  \cite{2}
\end{quote}
\textbf{proof:}
\begin{quote}
In \cite{2} on page 274.
\end{quote}
\begin{thebibliography}{2}
\bibitem[GSS]{1} Golubitsky, Martin. Stewart, Ian. Schaeffer, G. David.: Singularities and Groups in Bifurcation Theory \textit{(Volume II)}. Springer-Verlag, New York, 1988.
\bibitem[HW]{2} Hermann, Weyl: The Classical Groups: Their Invariants and Representations. Princeton University Press, New Jersey, 1946.
\end{thebibliography}
%%%%%
%%%%%
\end{document}
