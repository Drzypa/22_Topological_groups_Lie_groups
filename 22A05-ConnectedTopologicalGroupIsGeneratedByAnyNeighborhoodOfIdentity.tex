\documentclass[12pt]{article}
\usepackage{pmmeta}
\pmcanonicalname{ConnectedTopologicalGroupIsGeneratedByAnyNeighborhoodOfIdentity}
\pmcreated{2013-03-22 18:01:45}
\pmmodified{2013-03-22 18:01:45}
\pmowner{asteroid}{17536}
\pmmodifier{asteroid}{17536}
\pmtitle{connected topological group is generated by any neighborhood of identity}
\pmrecord{7}{40547}
\pmprivacy{1}
\pmauthor{asteroid}{17536}
\pmtype{Theorem}
\pmcomment{trigger rebuild}
\pmclassification{msc}{22A05}

% this is the default PlanetMath preamble.  as your knowledge
% of TeX increases, you will probably want to edit this, but
% it should be fine as is for beginners.

% almost certainly you want these
\usepackage{amssymb}
\usepackage{amsmath}
\usepackage{amsfonts}

% used for TeXing text within eps files
%\usepackage{psfrag}
% need this for including graphics (\includegraphics)
%\usepackage{graphicx}
% for neatly defining theorems and propositions
%\usepackage{amsthm}
% making logically defined graphics
%%%\usepackage{xypic}

% there are many more packages, add them here as you need them

% define commands here

\begin{document}
{\bf Theorem -} Let $G$ be a connected topological group and $e$ its identity element. If $U$ is any open neighborhood of $e$, then $G$ is generated by $U$.

$\,$

{\bf \emph{Proof:}} Let $U$ be an open neighborhood of $e$. For each $n \in \mathbb{N}$ we denote by $U^n$ the set of elements of the form $u_1 \dots u_n$, where each $u_i \in U$. Let $W := \bigcup_{n \in \mathbb{N}} U^n$.

Since each $U^n$ is open (see \PMlinkname{this entry}{BasicResultsInTopologicalGroups} - \PMlinkescapetext{Proposition} 3), we have that $W$ is an open set. We now see that it is also closed.

Let $g \in \overline{W}$, the closure of $W$. Since $gU^{-1}$ is an open neighborhood of $g$, it must intersect $W$. Thus, let $h \in W \cap gU^{-1}$.
\begin{itemize}
\item Since $h \in gU^{-1}$, then $h = gu^{-1}$ for some element $u \in U$.
\item Since $h \in W$, then $h \in U^n$ for some $n \in \mathbb{N}$, i.e. $h=u_1 \dots u_n$ with each $u_i \in U$.
\end{itemize}
We then have $g= u_1 \dots u_n u$, i.e. $g \in U^{n+1} \subseteq W$. Hence, $W$ is closed.

Since $G$ is connected and $W$ is open and closed, we must have $W = G$. This means that $G$ is generated by $U$. $\square$
%%%%%
%%%%%
\end{document}
