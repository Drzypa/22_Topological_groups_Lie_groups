\documentclass[12pt]{article}
\usepackage{pmmeta}
\pmcanonicalname{CompactGroupsAreUnimodular}
\pmcreated{2013-03-22 17:58:23}
\pmmodified{2013-03-22 17:58:23}
\pmowner{asteroid}{17536}
\pmmodifier{asteroid}{17536}
\pmtitle{compact groups are unimodular}
\pmrecord{4}{40481}
\pmprivacy{1}
\pmauthor{asteroid}{17536}
\pmtype{Theorem}
\pmcomment{trigger rebuild}
\pmclassification{msc}{22C05}
\pmclassification{msc}{28C10}

\endmetadata

% this is the default PlanetMath preamble.  as your knowledge
% of TeX increases, you will probably want to edit this, but
% it should be fine as is for beginners.

% almost certainly you want these
\usepackage{amssymb}
\usepackage{amsmath}
\usepackage{amsfonts}

% used for TeXing text within eps files
%\usepackage{psfrag}
% need this for including graphics (\includegraphics)
%\usepackage{graphicx}
% for neatly defining theorems and propositions
%\usepackage{amsthm}
% making logically defined graphics
%%%\usepackage{xypic}

% there are many more packages, add them here as you need them

% define commands here

\begin{document}
{\bf Theorem -} If $G$ is a compact Hausdorff topological group, then $G$ is unimodular, i.e. it's left and right Haar measures coincide.

$\,$

{\bf \emph{Proof}:}

Let $\Delta$ denote the modular function of $G$. It is enough to prove that $\Delta$ is constant and equal to $1$, since this proves that every left Haar measure is right invariant.

Since $\Delta$ is continuous and $G$ is compact, $\Delta(G)$ is a compact subset of $\mathbb{R}^+$. In particular, $\Delta(G)$ is a bounded subset of $\mathbb{R}^+$.

But if $\Delta$ is not identically one, then there is a $t \in G$ such that $\Delta(t) >1$ (recall that $\Delta$ is an homomorphism). Hence, $\Delta(t^n) = \Delta(t)^n \longrightarrow \infty$ as $n \in \mathbb{N}$ increases, which is a contradiction since $\Delta(G)$ is bounded. $\square$

%%%%%
%%%%%
\end{document}
